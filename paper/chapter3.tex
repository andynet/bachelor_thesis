\chapter{Data retrieval and preprocessing}
In this chapter we will present parts of our pipeline intended for data retrieval and preprocessing.
Since we used a considerable amount of third party libraries and tools, we will explain them in detail and we will clarify our reasons to use them.
We will also provide some basic statistics of data retrieved.

\section{Pipeline overview}
Our pipeline consists of python scripts and third party bioinformatics software.
When writing the code, we have taken care of its readability, sustainability and extensibility.

As the main script we used Snakefile written in Snakemake.
The Snakemake is a workflow management system designed for writing reproducible bioinformatics pipelines.
It is inspired by GNU make, but it use python-like syntax with elements similar to pseudo code.
Furthermore, it is fully portable, with dependency only on python.
Snakefile consists of rules, where each rule is defined by its input files, output files and shell commands.
These are commands needed to execute in order to produce output files from input files.
When the Snakemake is executed, by default, it runs first rule in particular Snakefile. 
If the rule cannot be completed, because of missing input files, it scan through the whole Snakefile and look for a rule, which is capable of creating desired file. 
This process is repeated until there is rule which can be completed or rule whose input is not possible to create by any other rule.
In the former case the execution starts running, in the latter case an error message is displayed.
By this approach it is ensured we do not run any unnecessary rules and already completed rules.
This is an important feature for our program as some rules can take several hours to complete even on a powerful server. 
Another pleasant characteristics of the Snakemake engine is the ability to produce graphical visualisation of particular Snakefile in format of directed acyclic graph.
Simplified graphical representation of our pipeline is in the Figure (\ref{fig:dag}).

The pipeline starts with downloading of publicly available data.
After downloading we merge all records and eliminate duplicated records.
Later on, we predict bacteriophages genes.
Consequently, phage genomes with their corresponding genes are split into training and testing set.
Similarity between genes from training set are calculated and based on those, clusters of similar genes are produced. 
From these clusters the binary matrix is created.
This matrix is later used in analysis as input to machine learning algorithms.

\begin{figure}[h]
\includegraphics[height=\textheight]{./images/mcl.png}
\centering
\caption{Workflow visualization}
\label{fig:dag}
\end{figure}

\section{Downloading}
Since the first step in our pipeline is downloading of data from publicly available databases, we will provide more detailed description of these sources.
In our pipeline we implemented downloading from three databases.
Although they cover the most of currently sequenced and published phages, we made this step easily extensible for new sources of information.
New source can be added by writing a new download script, naming it \verb|{script_dir}\download_from_{db}.py| and appending this name into variable \verb|DATABASES| in Snakefile. 

\subsection{Downloading from GenBank}
National Center for Biotechnology Information (NCBI) provides an access to GenBank\cite{genbank} database.
This database is a comprehensive source of genomic data with more than 200 million sequences.
Sequences are primarily submitted by individuals all around the globe.
Moreover, GenBank is daily synchronized with European Nucleotide Archive and DNA Data Bank of Japan, which ensures that data are always up-to-date with human knowledge.
NCBI administer GenBank database free of charge and give researchers the possibility to access data through various interfaces as web-based retrieval services, FTP and Entrez\cite{entrez}.
Despite of these facts, there are also shortcomings of using GenBank.
In the time of Next Generation Sequencing the amount of data flowing into GenBank database every day is enormous.
Therefore it is unreasonable to check all data.
This is the cause of redundancy of sequences and sometimes contradictions between information in system.

In our work, data from GenBank was downloaded through our custom script.
We obtained data using python library Biopython \cite{biopython}, which implements python wrapper NCBI Entrez.
Besides, Biopython was also useful for its classes enabling easy manipulation with standard file formats used in bioinformatics.
When downloading sequences, we also created unique identifiers for each record.
Those were used later in pipeline.
Reasons behind the decision to use custom identifiers were ability to remove duplicated sequences and ability to find out possibly multiple sources of each sequence in our dataset.
Downloading from GenBank was our largest source of data with 6704 downloaded records.

\subsection{Downloading from ViralZone}
ViralZone is a website dedicated to viruses. 
It provides highly reliable data about viruses, including bacteriophages.
Information about structure of capsid and genome, life cycle, replication mechanisms, taxonomy, geographical location and host are included.
This website does not store sequences internally, rather it delivers links to RefSeq\cite{refseq} database.
Compared to GenBank, RefSeq database contains fewer sequences, but all of these sequences are curated and manually reviewed.

Our custom script was used to download records from RefSeq database.
Although, large portion of sequences downloaded was identical with GenBank records, some sequences were unique.
Another advantage in performing this action was that it enabled us to pair records in our dataset with full information from ViralZone portal.
By this process we obtained 2107 records. 

% end of session 2018.04.18
% wc chapter* = 32521 letters

\subsection{Downloading from PhagesDB}
PhagesDB is a database specialized for bacteriophages infecting bacteria from phylum Actinobacteria.
This phylum is of great importance, because of its contribution to soil system.
Phylum also contains genus Mycobacterium, which includes pathogens causing tuberculosis and leprosy in humans.
The database was designed to avoid the time between sequencing and data availability.
Authors declare, at the time of their publication, there was more than 600 records of bacteriophages that were not yet in GenBank.
Furthermore, PhagesDB stores more biologically relevant data which are easily obtainable through its Application Programming Interface (API).
These biological data contains discovery details, sequencing details, characterization details, sequence file and plaque picture.
From this database we downloaded 2491 phage records with our automatized script using API.

\section{Merging and removing of duplicated records}
We were aware of high redundancy in our datasets, mainly because many of those records was submitted to more databases at once.
To solve this issue, we merged datasets together and removed duplicated records.
For merging we used standard unix command \verb|cat|.
For removing duplicated sequences we created python script.
This script made use of custom identifiers, which were assigned to every sequence downloaded.
If more identical sequences were found their identifiers were rewritten with identifier of the first sequence.
This approach preserved the relationships between one particular sequence and all data related to it.
Therefore we were able to track phages, based on their identifiers, to their source databases and also connect them with all data already downloaded.
After removing of duplicated sequences our dataset contained 6277 phage records.

\section{Gene identification}
Next step in the pipeline was to identify genes in the dataset.
Although gene annotations of particular genomes can be found in most databases, we decided to annotate sequences from scratch.
This decision was based on our request for consistency of predicted and annotated genes.
Another advantage of annotating from scratch is that annotations will always be up-to-date with current human knowledge.

\subsection{Prokka}
For the purpose of gene identification we used third party software Prokka\cite{prokka}.
Prokka uses external tools for identification and annotation of genes.
First, coordinates of coding DNA sequences (CDS) are found with Prodigal tool\cite{prodigal}.
CDSs are regions of genome that almost always start with AUG codon and end with stop codon.
These regions can be directly translated into amino acid chains using standard codon table.
After locations of genes were predicted, Prokka starts to annotate functions of all CDSs.
This is usually done through comparing of sequence to database of sequences with experimentally determined function.
The function of protein with the best match is than assigned to new CDS.
Prokka also uses this approach, but it searches through multiple databases.
Starting with the most reliable source, which is usually the smallest, it scans all the databases, always continuing with the less accurate one.
The databases used with their corresponding order as as follows:
An optional user-defined database, UniProt\cite{uniprot}, RefSeq\cite{refseq}, Pfam\cite{pfam} and TIGRFAM\cite{tigrfam}.
If no match is found across databases, protein is labelled as \verb|hypothetical protein|. 

% end of session 2018.04.12
% wc chapter* = 33657 letters

\subsubsection{UniProt}
The UniProt database is a huge collection of protein sequences and their corresponding detailed information.
It contains more than 60 millions of proteins.
Prokka uses just a small fraction of proteins backed up by experimental evidences.
This typically provide information about around 50\% of queried proteins.

\subsubsection{RefSeq}
RefSeq, in addition to genomic sequences, provides also information about protein sequences.
It contains more than 2.5 millions of protein records.
Multiple sources are integrated in annotation of genes.
Furthermore, all records are curated by NCBI staff members.

\subsubsection{Pfam}
Pfam is a database consisting of protein families and domain records.
Protein families are groups of protein sharing evolutionary history.
This shared evolutionary history is often expressed by closely related functions and sequence similarity.
Protein domains are parts of protein sequence responsible for a particular interaction.
Members of the same protein domain usually share high sequence similarity.
Protein families and domains are mostly characterized by Profile Hidden Markov Models.
Pfam incorporate more than 6100 of those models.

\subsubsection{TIGRFAM}
TIGRFAM, similarly as Pfam, contains protein families characterized by Profile Hidden Markov Models.
It contains more than 4200 models with comprehensive description of family structure and function.

\vspace{\baselineskip}

All of the aforementioned databases are regularly updated, which guarantee the most up-to-data annotations. 
After annotation, resulting genes were selected and saved to files in format suitable for further use in the pipeline.

\section{Training set, testing set and other}
At the beginning of this step our dataset consisted of phage genomic records, their corresponding genes, information about phage hosts and functional annotation of genes.
As our work used techniques of supervised machine learning, we needed to split the dataset to training set and testing set.
Furthermore, we created set for other records, which we decided not to use due to missing information about host or due to host outside of our group of interest.
We decided to group phages according to genus of their hosts.
After calculating number of phages in each group, we selected first eight genera with the highest count of records as groups of our interest.
These were Mycobacterium, Streptococcus, Escherichia, Gordonia, Arthrobacter, Pseudomonas, Lactococcus and Staphylococcus. 
Number of phages within each group can be found in the Table (\ref{tab:counts}).
All other genera were excluded from the dataset due to insufficient amount of samples.
Phages without information about their hosts were also excluded.
Subsequently, we divided remaining data into training set and testing set at a ratio 4:1.
Resulting training set included 2787 records of bacteriophages and resulting testing set included 699 records. 

\begin{table}
  \centering
    \begin{tabular}{ l  r  l  r }
      \hline
      genera & count & genera & count \\
      \hline
      Mycobacterium & 1619 & Streptococcus & 354 \\
      Escherichia & 323 & Gordonia & 293 \\
      Arthrobacter & 240 & Pseudomonas & 236 \\
      Lactococcus & 219 & Staphylococcus & 184 \\
      \hline
    \end{tabular}
    \caption{Counts of records within genera}
    \label{tab:counts}
\end{table}

%--------deeply described parts-------------------------------------------------
\section{Sequence alignment}
In this section we would like to deeply explain what is sequence alignment as it is one of the most essential task in bioinformatics.
Because of its usage in some form in almost all task related to biological sequences, we will describe possibilities of its application.
As there are two main types of sequence alignment, we will describe both of them with emphasizing the differences.
We will also go through the basic algorithm used for solving these problems.
At the end of this section we will describe, how we used them in our analysis.

\subsection{Global alignment}

\subsubsection{Usage}
Global alignment is commonly used from comparison between sequences from their beginning to their end.
This comparison can help us to discover mutations in sequence, which cause certain phenotype or to determine highly conserved regions with a potential to carry a gene.
The relationships between conserved regions and mutations in certain sequences often serves as a basic assumptions for construction of phylogenetic trees.
Therefore, it can tell us a lot about the nature of evolution.
Global alignment is also capable of calculating similarity score for two sequences.
In general, it is mostly used when comparing sequences with roughly the same length.

\subsubsection{Problem statement}
Let the input for the problem be set of two sequences consisting of nucleotides.
For example, $ X = ATTGATGG $ and $ Y = AATTCAAC $.
Then the output is represented as matrix, where each row is one sequence with possible gaps between nucleotides.
We can see potential solutions in the Table (\ref{tab:potsol})

\begin{table}
  \centering
    \begin{tabular}{ l | r }
    \verb|A-TTGATGG| & \verb|-ATTG-ATGG| \\
    \verb|AATTCAAC-| & \verb|AATTCAAC--| \\
    \end{tabular}
  \caption{Global alignment examples}
  \label{tab:potsol}
\end{table} 

As we can see, every alignment could have more than one valid solution.
Despite of multiple possible solutions, not every solution is equally good.
Typically, we want to find the best possible solution to this problem.
For this purpose, we use scoring scheme.
Scoring scheme consists of rules, which add numerical value to each column of pairwise alignment.
For example, we can evaluate match in column with score $+1$, mismatch with score $-1$ and alignment to gap with score $-1$.
With this scoring scheme, we can evaluate quality of particular alignment.
For alignment on the left side of the Table (\ref{tab:potsol}), resulting score is $+1-1+1+1-1+1-1-1-1 = -1$ and for alignment on the right side of the Table (\ref{tab:potsol}) $-1+1+1+1-1-1+1-1-1-1 = -2$.
From this example, we can clearly see, that according to our scoring scheme alignment on the left is better.

\subsubsection{Scoring}
In practice we could use more complex scoring scheme, which better reflects reality.
For example substitution between purines (adenine, guanine) or substitution between pyrimidines (cytosine, thymine) occur more often, because they do not require change in the number of rings in the chemical structure of these nucleotides.
Another example, when more complex scoring scheme is inevitable is aligning of two proteins.
Amino acids differ in many parameters as polarity, size and structure of their side chains.
This influences the probability of substitution between two amino acids.
Therefore, it is much more likely that Leucine will be substituted for Isoleucine, than substituted for Aspartic Acid.
To solve the complexity of substitutions between amino acids, matrix BLOSUM62 (\ref{fig:blosum}) was created.
Another layer of complexity, which reflects reality better, is affine gap penalty function.
This reflects the fact that insertions and deletions do not usually occur only on one nucleotide, but often a longer region of DNA is deleted or inserted.
Affine gap penalty solves this issue by higher negative score for opening a new gap in alignment and a lower negative gap for extension of already created gap.

\begin{figure}[ht]
  \centering
    \includegraphics[width=\textwidth]{./images/blosum62.png}
  \caption{BLOSUM62}
  \label{fig:blosum}
\end{figure}

% end of session 16.04.2018

\subsubsection{Needleman-Wunsch algorithm}
For searching optimal global alignment we usually use Needleman-Wunsch algorithm.
It is an algorithm from group of dynamic programming algorithms.
This means, the main problem is divided into smaller problems, which are computable more easily and solutions to them are stored in memory.
At each occurrence of a small problem, we can look into stored solutions, where we can find it. 
Then the main problem is reconstructed from already computed subproblems.
Dynamic programming algorithms offer saving time on computation at the expense of higher memory usage.

\begin{figure}[ht]
  \centering
    \includegraphics[width=\textwidth]{./images/needle_wunsch.png}
  \caption{Needleman-Wunsch}
  \label{fig:glal}
\end{figure}

Needleman-Wunsch algorithm produces table as in the Figure (\ref{fig:glal}). 
It starts by putting the first sequence we want to align to first row and the second sequence to first column.
Before each sequence there is one gap to cover the case if we would not want to align first letter of particular sequence right from beginning.
The table is then initialized with series starting from 0 and decreasing by 1 each step on the second row and with the same series on the second column.
After initialization the table starts to be filled from top left corner following this rule:\\
Into each cell $A_{i,j}$ write maximum of:
\begin{itemize}
\item $A_{i-1, j-1} + s()$,
\item $A_{i-1, j} + g()$,
\item $A_{i, j-1} + g()$,
\end{itemize}
where s() returns score of match/mismatch from scoring scheme and g() returns score of gap penalty (possibly affine gap).
The final score of the alignment can be found in the bottom right corner of filled table.
Specific alignments can be found by tracking all possible paths to this value.
The time and space complexity of Needleman-Wunsch algorithm is $\mathcal{O}(nm)$, where $n$ is the length of the first sequence and $m$ is the length of the second sequence.

\subsection{Local alignment}
\subsubsection{Usage}
In comparison with global alignment, local alignment search for regions inside sequences with high similarities and does not provide alignment from beginning to end.
It is generally useful when searching for a small subsequence inside vast sequence.
For example, searching for a gene inside whole bacterial genome.
It can be also used when comparing two different sequences and we want to find out if they contain any highly similar sequence.

\subsubsection{Problem statement}
Similarly as in global alignment, in this problem we search for optimal local alignment according to defined scoring system.
The difference is that we do not know where the alignment in both sequences starts and where it ends.
For example, $ X = TAATAACTCTCTGAATAA $ and $ Y = CGGCGGCGGTCTCTGCC $ can be aligned as in the Figure (\ref{tab:loal}) and the score is calculated just from the first aligned base to the last aligned base.

\begin{table}
  \centering
    \begin{tabular}{ c }
    \verb|--TAATAACTCTCTGAATAA| \\
    \verb%         ||||||     % \\
    \verb|CGGCGGCGGTCTCTGCC---| \\
    \end{tabular}
  \caption{Local alignment example}
  \label{tab:loal}
\end{table} 

\subsubsection{Scoring}
Scoring of local alignment is also similar to global alignment and we are allowed to use the same methods as in global alignment.
Since the score is calculated just from the first aligned base to the last aligned base, the score of our local alignment would be $+1+1+1+1+1+1 = 6$, because there are 6 matches ($+1$) and no gaps ($-1$) or mismatches ($-1$).

\subsubsection{Smith-Waterman algorithm}
Local alignment can be found with dynamic programming algorithm similar to Needleman-Wunsch algorithm.
This is called Smith-Waterman algorithm and there are only two changes compared to Needleman-Wunsch.
First change is that matrix is not initialized with decreasing series, but second row and second column are filled with zeroes.
Second difference is in the rule as follows:\\
Into each cell $A_{i,j}$ write maximum of:
\begin{itemize}
\item $0$,
\item $A_{i-1, j-1} + s()$,
\item $A_{i-1, j} + g()$,
\item $A_{i, j-1} + g()$.
\end{itemize}
After completing the table, we need to find the highest number in it.
This is resulting score of our local alignment.
Following the path similarly as in global alignment we can reconstruct the alignment.
The space and time complexity of this algorithm is also $\mathcal{O}(nm)$.
 
\subsection{Word methods}
Although Needleman-Wunsch and Smith-Waterman algorithms are sufficient for simple comparisons of sequences, their time complexity is not good enough when we want to search through big data.
It is often the case, that we want to find the most similar sequence to ours in enormous bioinformatics database containing several completed genomes.
This is particularly useful if we want to find potential source of our sequence or comparing it to all known proteins to get some indicators about its potential function.
For this purpose various heuristic algorithms were developed.
These algorithms do not guarantee finding the most optimal solution, but are orders of magnitude faster and therefore usable also for searching in vast databases.
  
\subsubsection{BLAST}
Basic Local Alignment Search Tool (BLAST) is probably the most widely used tool in bioinformatics.
% end of session 23.04.2018  
 
\subsubsection{E-value}

\subsection{Alignment in the pipeline}
After dataset splitting we performed local alignment of the genes in training set against themselves.
Thanks to this step we were able to quantify similarities between different genes, which we needed further in pipeline at clustering step.
The standard bioinformatics tool for this purpose is BLAST \cite{blast}. 
The basic algorithm consists of searching seeds from database on the query sequence and then extending those seeds into neighbouring bases.
This approach provides orders of magnitude faster alignment method than classical Smith-Waterman algorithm \cite{smith_waterman} with comparable sensitivity.
In our pipeline we used software CrocoBLAST \cite{crocoblast}, what is a wrapper around BLAST algorithm which make better use of paralellization than standard BLAST maintained by NCBI.
With CrocoBLAST we were able to reduce time of this local alignment step from four days to one day as opposed to BLAST.
Resulting file was in tab separated format, where first column was gene identifier of query sequence, second column was gene identifier of the target sequence and third column was e-value of alignment. 
E-value is evaluation measure of how similar two sequences are.

\section{Data clustering}
In this section we will explain what is data clustering.
We will describe its use in data analysis and bioinformatics and how we used it in our program.
We will briefly define the problem and we will explain algorithm used for solving it.
At the end we will present three approaches used in our pipeline.

\subsection{Usage}
Data clustering is used to group data according to some similar characteristics.
In computational biology it can be used to cluster organisms in order to examine relationships between different groups or to study structure of population.
By clustering genes we can investigate their functions and see which regions in sequence are responsible for them.
Clustering can also be used to distinguish different types of tissue or group drugs with similar effects according to their mechanism of action.

In our work, we clustered genes with sequential similarities.
We expected, that this approach will create clusters of genes with closely related functions.

\subsection{Definition of problem}
Clustering is a problem where we want to determine closely related entities and put them in distinct groups, also called clusters.
Cluster can be characterized by presence of many edges between cluster members.
Members from different clusters are connected with lesser probability and the distance between them is usually longer.
Although, to intuitively understand what is clustering is not hard, there exists a lot of clustering models with slightly different approaches, which makes exact general definition impossible.  

\subsection{Markov Cluster Algorithm}
Markov Cluster Algorithm is algorithm searching for clusters by simulating random walks on the graph.
It assumes that these random walks will infrequently go from one cluster to different cluster and mostly will stay in the starting cluster.
The algorithm simulates random walks through the graph deterministically with two operations, expansion and inflation.
Expansion corresponds with squaring of stochastic matrix, non-negative matrix where each column sums to 1 and entry $a_{ij}$ corresponds to probability of going from node $j$ to node $i$.
Inflation is defined by entry-wise powering of stochastic matrix to the power of $r$, followed by scaling step to get stochastic matrix again.
It is observed that for $r>1$ inflation favour more probable walks over less probable walks.
Expansion can be interpreted as random walk with many steps.
Inflation then enlarge the effect of random walks within cluster and reduce random walks across different cluster.
By repeating expansion and inflation operations, the equilibrium is reached which produce clusters.
These clusters does not contain any paths between each other and the only paths in graph are those within clusters.
This algorithm does not need any prior knowledge of clusters and resulting clusters emerge from the primary structure of the graph.
Furthermore, size of clusters can be regulated by changing parameter $r$ of inflation operation.
Bigger $r$ will cause algorithm to converge to equilibrium faster and will produce smaller resulting clusters.
In practice, $r$ from 1.2 to 5.0 is used.

% \subsection{Spectral clustering}
% http://www.paccanarolab.org/scps/

\subsection{Clustering in our pipeline}
\subsubsection{MCL}
Firstly, we used Markov Cluster Algorithm implemented in package MCL \cite{mcl}. 
This software is popular among bioinformatics community for its capabilities to work with big data.
As input data we used E-values from CrocoBLAST results.
These E-values was automatically transformed into similarity score to enable creation of adjacency matrix.
As inflation parameter we used value 1.2.
This was the smallest value recommended by developers.
The reason for using the smallest possible inflation value was, that we wanted to create the biggest possible clusters.
This approach also reduced number of different clusters.
In our work we needed as few clusters as possible, mainly due to number of phage records in our dataset.
In case of too many clusters we would have too many features for classifier and we would risk overfitting of our final models to training dataset.
Finally, we created 15017 gene clusters.

\subsubsection{MCL with global alignment}
To get more accurate clustering we tried different approach, where we determined similarity score from global alignment.
All the matches from CrocoBLAST search were aligned with needleman-wunsch algorithm \cite{needleman-wunsch} and the resulting score of alignment was used instead of E-value.
Implementation of Markov Cluster Algorithm was executed with inflation parameter of value 1.2, without any transformation of similarity score.
By this approach we created 9176 final clusters.

\subsubsection{Spectral clustering}
In this work, we also experimented with different clustering algorithm, Spectral clustering.
SCPS implementation\cite{scps} was tested in the pipeline.
Although authors of this software declare quality of clusters quantified by a measure that combines sensitivity and specificity to be better by 28\% in comparison to MCL algorithm, our memory was not sufficient for the program to run with all input data.

%--------end of deeply described parts------------------------------------------

\section{Cluster annotation}
Resulting clusters were annotated for their function.
We expected proteins with similar function to be included in same cluster.
For functional annotation of particular proteins we used software InterProScan\cite{interpro}.
This tool scans given protein sequences against the protein signatures in databases PROSITE\cite{prosite}, PRINTS\cite{prints}, Pfam\cite{pfam}, ProDom\cite{prodom} and SMART\cite{smart}.
After acquiring of annotations for all proteins in particular cluster, we calculated number of occurrences of each protein function.
These statistics represents our annotation of certain cluster.
Our expectation of clusters containing proteins with similar function was met in most cases, although the information about particular proteins were sparse with a lot of proteins without any information.
Therefore, reasonable clustering was achieved and is supported by our knowledge of phage biology.

\section{Creation of matrix}
One of the most crucial part of our analysis was binary matrix created in this step.
Rows in this matrix represented particular phage record and columns represented particular protein cluster.
The entry $a_{i,j}$ in matrix was filled with $1$ if phage $i$ contained gene from cluster $j$ and $0$ otherwise.
Custom python script and files produced in previous steps were used for this task.
Resulting matrix contained 2787 rows and 15017 columns and served as a main input file for machine learning algorithms.


