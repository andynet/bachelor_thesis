\chapter{Data retrieval and preprocessing}
In this chapter we will present our pipeline intended for data retrieval and preprocessing.
Since we used a considerable amount of third party libraries and tools, we will explain them in detail and we will clarify our reasons to use them.
We will also provide some basic statistics of data retrieved.

\section{Brief pipeline overview}
Our pipeline consists of python scripts and third party bioinformatics software.
When writing the code, we have taken care of its readability, sustainability and extensibility.
The time of execution of bioinformatics programs can be enormous, so we divided our pipeline to various steps.
After each step, there is a checkpoint, which ensures we will not need to run this step again in case of unexpected crash of pipeline.
The pipeline starts with downloading of publicly available data from 3 sources - NCBI, viralzone and phagesdb.
After downloading we merge all records into one file and eliminate duplicated records.
Later on we predict bacteriophages genes with Prokka \cite{prokka}.
Next step was to cluster all predicted genes into clusters based on similarity of protein sequences.
For this purpose we use CrocoBLAST \cite{crocoblast}, EMBOSS needle software \cite{needle} and mcl algorithm \cite{mcl}.
After clustering we created matrix, where rows were individual bacteriophages and columns were our clusters.
In this matrix 1 represented, that given phage contains a gene from given cluster and 0 represented it does not contain a gene from a given cluster.
This matrix was later used in analysis as input to machine learning algorithms.

\section{Snakefile}
Snakefile is the main script, which executes all other components of our program. 
It is written in Snakemake \cite{snakemake}, workflow engine designed for writing reproducible pipelines in bioinformatics. 
Snakemake is inspired by GNU make, but it use python-like syntax with elements similar to pseudo code.
Furthermore, it is fully portable, with dependency only on python.
Snakefile consists of rules, where each rule is defined by its input files, output files and shell commands.
These are commands needed to execute to produce output files from input files.
When the Snakemake is executed, by default, it runs first rule in particular Snakefile. 
If the rule cannot be completed, because of missing input files, it scan through the whole Snakefile and look for a rule, which is capable of creating desired file. 
This process is repeated until there is rule which can be completed or rule whose input is not possible to create by any other rule.
By this approach it is ensured we do not run any unnecessary rules and already completed rules.
This is an important feature for our work as some rules can take several hours to complete even on a powerful server. 

\section{Downloading}
First rule to complete in our workflow was downloading data from publicly available databases.
We made this rule easily extensible for new sources of information.
New database can be added by writing a new download script, 
naming it \verb|{script_dir}\download_from_{db}.py| and appending this name into variable \verb|DATABASES| in config file. We implemented downloading from 3 sources - NCBI \cite{ncbi}, viralzone \cite{viralzone}, phagesdb  \cite{phagesdb}.

\subsection{Downloading from NCBI}
NCBI, National Center for Biotechnology Information, was the largest source of data for our pipeline.
Their nucleotide database should contain all published genomes.
Downloading of these genomes from their database is possible through web interface or through API.
In our script we used python library Biopython \cite{biopython}, which provides python wrapper around NCBI API.
Besides, Biopython offers useful classes for work with standard file formats used in bioinformatics.
Downloaded data were stored in four files with extensions genomes.fasta, genomes.conversion, genes.fasta, genes.conversion.
Fasta files were described in chapter 1.
Conversion files are tab separated files containing metadata about particular downloaded sequences.
In file with extension genomes.conversion these metadata are phage identifier in our database, phage identifier in external database, phage name and its hosts.
In file with extension genes.conversion these metadata are gene identifier in our database, our phage identifier of bacteriophage with this gene, identifier of gene from external database, its position in phage genome and function.
Reasons for creation of our internal identifiers were inconsistency in naming conventions between different databases.
Our software downloaded all records in NCBI nucleotide database satisfying query 
\begin{verbatim}
'phage[Title] AND (complete genome[Title] 
OR complete sequence[Title]) AND (viruses[filter] 
AND biomol_genomic[PROP] AND ("10000"[SLEN] : "100000000"[SLEN]))'
\end{verbatim}. There was 6704 such records. 

\subsection{Downloading from viralzone}
Viralzone is a website dedicated to viruses. 
It provides high quality data about viruses, including bacteriophages.
We used this website to get query similar to query in previous section.
This new query was used by the same program as our query.
There was 2107 records satisfying query from viralzone. (18-03-08)

\subsection{Downloading from PhagesDB}
PhagesDB is an independent database from NCBI. 
This database is specialized for Actinobacteriophages, so it has less phage records, but it is designed to avoid the time between sequencing and data availability.
Authors declare, at the time of their publication, there was more than 600 records of bacteriophages in phagesdb that were not yet in GenBank.

Furthermore, PhagesDB stores more biological relevant data which are easily obtainable through its RESTful Application Programming Interface.
In our work we used this API to retrieve all genomic sequences with their hosts and annotations.
Eventually we ended up with 2491 phage records from PhagesDB 

\section{Merging and removing of duplicated records}
All download scripts produced four types of output - \verb|<db>.genomes.fasta| \verb|<db>.genomes.conversion| \verb|<db>.genes.fasta| \verb|<db>.genes.conversion|. 
We were aware of high redundancy in our datasets, mainly because many of these records was submitted to more databases at once.
To solve this issue, we merged datasets together and removed duplicated records.
For merging we used standard unix command \verb|cat| for all files with the same type.
For removing duplicated we created our python script.
We looked on all fasta records in merged \verb|genomes.fasta| and if there were more than one with the same sequence, we wrote the one with lesser internal identifier into output fasta file.
We also changed other internal identifiers in files \verb|genomes.conversion| and \verb|genes.conversion| into this identifier.
This approach preserves the relationships between one particular sequence and all data related to it.
Therefore we are able to track phages, based on their internal identifier, to their source databases and also connect them with data already downloaded.
After duplicate removing our dataset consisted of 6277 phage records.

\section{Gene identification}
Despite of downloading annotated sequences with genes, we do not use those genes further in out workflow.
We made this decision because the inconsistency of gene annotations between different databases and different records could be potentially high.
To ensure predictions will be consistent along the entire dataset, we predicted all genes and their annotations by one software - Prokka.
In addition, this software uses regularly updated databases, which guarantee the most up-to-date annotations.
After annotation we extracted all genes with our script.
The output of this program were files \verb|annotated.genes.fasta| and \verb|annotated.genes.conversion| which were further used in pipeline.

\section{Splitting to training set and testing set}
In order to properly evaluate our final models, we divided whole dataset into training set and testing set.
First we find out how many bacteriophages for every group of hosts we have. 
Names of this groups were abreviations of host names.
Reason to use abreviations as group names is that we wanted bring together phages with similar hosts in one group.
We also needed to deal with typos and different formats of the same host.
We selected first 8 groups with the highest counts of records.
These were mycobac, strepto, escheri, gordoni, arthrob, pseudom, lactoco and staphyl with counts 1619, 354, 323, 293, 240, 236, 219, 184 respectively.
All other groups were excluded from our dataset due to insufficient amount of samples.
Furthermore we excluded all phages without the information about their hosts.
This new purified dataset was divided into training set and testing set at a ratio 4:1.
At the end of this step the training set consisted of 2787 records of bacteriophages and the testing set consisted of 699 records.

\section{Alignments}
\section{Local alignment of genes}
After dataset splitting we performed local alignment of the genes in training set against themselves.
Thanks to this step we were able to quantify similarities between different genes, which we needed further in pipeline at clustering step.
The standard bioinformatics tool for this purpose is BLAST \cite{blast}. 
The basic algorithm consists of searching seeds from database on the query sequence and then extending those seeds into neighbouring bases.
This approach provides orders of magnitude faster alignment method than classical Smith-Waterman algorithm \cite{smith_waterman} with comparable sensitivity.
In our pipeline we used software CrocoBLAST \cite{crocoblast}, what is a wrapper around BLAST algorithm which make better use of paralellization than standard BLAST maintained by NCBI.
With CrocoBLAST we were able to reduce time of this local alignment step from four days to one day as opposed to BLAST.
Resulting file was in tab separated format, where first column was gene identifier of query sequence, second column was gene identifier of the target sequence and third column was e-value of alignment. 
E-value is evaluation measure of how similar two sequences are.

\section{Clustering}
Clustering is a problem where we want to determine closely related entities and put them in distinct groups, also called clusters.
In order to be able to distinguish which entity is more closely related to another, we usually use similarity matrix.
To create a similarity matrix we used tab separated file described in previous section.
In this section we will mention different approaches we used for clustering. 
% http://www.paccanarolab.org/scps/
\subsection{Markov Cluster Algorithm}
First approach we used was Markov Cluster Algorithm. 
We used already implemented version called mcl \cite{mcl}.
This software is often used in bioinformatics for clustering of sequences based on their similarity.
It has the capability to determine number of resulting clusters automatically and except input data, it requires only one parameter.
This parameter is called inflation and usually takes float number values from 1.2 to 5.0.
Smaller values results in less and bigger clusters.
In our work we needed as few clusters as possible, mainly due to number of phage records in our dataset.
In case of too many clusters we would have too many features for classifier and we would risk overfitting of our final models to training dataset.
For these reasons we used inflation value 1.2, which created 15017 clusters.

\subsection{Markov Cluster Algorithm with global alignment}
To get more accurate clustering we tried different approach, where we determined similarity score of alignments with needleman-wunsch algorithm. \cite{needleman-wunsch}
This algorithm is able to calculate best global alignment score of given protein sequences.
The difference between global alignment and local alignment is that former aligns sequences from end to end.
The resulting score was then used as input to mcl algorithm described in previous section.
By this approach we created 9176 final clusters.

\subsection{Spectral clustering}
Spectral clustering is different algorithm used for clustering.
In this work we tried SCPS implementation\cite{scps}.
Authors of this bioinformatics software declare quality of clusters quantified by a measure that combines sensitivity and specificity to be better by 28\% in comparison to mcl algorithm.
Unfortunately, memory required by this program was too high and we were not able to run it on our server with all input data. 

\section{Cluster annotation}
Created clusters were further annotated to examine their functions.
We expected that proteins with similar function will end up in the same cluster. 
This was mostly the case although the information about particular proteins were sparse with a lot of proteins without any information in database.
For annotation we used software interproscan\cite{interpro}.
This tool scans given protein sequences against the protein signatures in databases PROSITE, PRINTS, Pfam, ProDom and SMART.

\section{Creation of matrix}
Probably the most important part of our pipeline is binary matrix created from phage records and clusters.
For the purpose of matrix creation we wrote our own script.
This script takes as input tree files - \verb|annotated.genes.conversion|, \verb|genes.clstr| and \verb|genomes.list|.
The first file is needed in order to know which phage contains which particular gene.
The second file is file containing gene clusters in tab separated format where each line represents one cluster.
The last file is just a list of all phages which should be included in matrix.
This script is also parallelized to enable use of multiple computational nodes.
We created raw matrix with 2787 rows representing phage records and 15017 columns represented gene clusters.
This matrix was further reduced with feature selection method Variance Threshold.
During this process we removed all columns with ones or zeroes in more than 99\%.
The removed columns have a high potential to be non-functional proteins which created its own clusters.
After feature selection our matrix contained 2787 phage records and 1818 distinct gene clusters.
This matrix was used as input to machine learning algorithms. 


