\chapter{Biological background}
In this chapter we will present some basic biological terms, which we will need later in this work.
At first we will describe the structure of DNA, its function in living organisms
and methods of representation in bioinformatics and computational biology.
We will also briefly describe how we can get information about DNA from living organisms and we will mention central dogma of molecular biology
% and we will provide more detailed description of bacteriophages,
% which will be the central topic of our work. At the end of this chapter we will point out what was already done in this field
%and what we would like to achieve.

\section{What is DNA}
Deoxyribonucleic acid \(DNA\) is a natural polymer, a long chain of nucleotides. There are 4 types of nucleotides
naturally occurring in DNA - adenine, cytosine, guanine and thymine. These bases are connected through phosphodiester bonds,
connection between the 3' carbon atom of one deoxyribose and the 5' carbon atom of second deoxyribose. \\
The structure of DNA consists of 2 complementary strands coiled around each other in the form of double helix.
In this double helix adenine is paired with thymine and cytosine with guanine through hydrogen bonds.
In bioinformatics and computational biology we usually represent bases by its first letter - A for adenine, C for cytosine, G for guanine and T for thymine.
Files in which we usually represent DNA sequences are simple plain text files. Probably the most widely used format is FASTA format.
It can consist of one or multiple records, where each record starts with one line for record id and comment, which is followed by possibly multiple lines of sequence writen in direction from 5' to 3'.

\section{DNA sequencing}
DNA sequencing is a process of retrieving data from biological samples.
Data of DNA are usually retrieved in a small fragments on account of limitations of technology.
Currently there are 3 main technologies of DNA sequencing - Sanger sequencing, NGS sequencing and Nanopore sequencing.
They provide us with different parameters as length of read, error rate, throughput and price per base.
We show a table of comparison between these techniques.

\begin{center}
  \begin{tabular}{ | l | c | c | c | c | }
    \hline
    name & length of read & error rate & throughput per day & price per MB \\
    \hline
    Sanger & 1000bp & <2\% & 3MB & 4000\$ \\
    NGS & 2*250bp & <1\% & 30GB & <1\$ \\
    Nanopore & 100kbp & 15\% & 1GB & 1\$ \\
    \hline
  \end{tabular}
\end{center}

\section{Central dogma of molecular biology}
We already showed some basic concepts about what DNA is and how we can work with it.
In this section we would like to present how DNA is interpreted in living organisms.
Usually DNA is transcripted into RNA molecules.
RNA is, similarly as DNA, a long chain of nucleotides.
The differences in these 2 molecules are, that RNA is mostly found in nature as single stranded molecule.
Nucleotides in RNA also contain ribose instead of deoxyribose and nucleotide thymine is altered for uracil.
RNA sometimes also exhibits enzymatic activity, but mostly it is translated into protein.
Proteins are long chains of amino acids and can be presented in computer in similar way as DNA or RNA, althought there is 20 proteinogenic amino acids.
We provide a table of codes for different aminoacids.
Proteins have highly variable functions and therefore they will be in the middle of our interest in this work.

\chapter{Bacteriophages}
In this chapter we will provide more detailed description of bacteriophages, which will be the central topic of our work.
We will describe what they are and where we can find them.
We will also show currently valid taxonomical classification and the structure of typical representant of this group of organism.
Next we will explain their life cycle and we will point out some of their use.

% At the end of this chapter we will point out what was already done in this field
% and what we would like to achieve.

\section{Origin of bacteriophages}
Bacteriophages belong to the group of viruses, which have the capability to infect and replicate within bacterial hosts.
It is estimated that they are also the most abundant group of entities on the earth with their estimated count around $10^{31}$.
In comparison, it is estimated the count of all bacteria on planet earth is around $10^{30}$.
They are also one of the most highly diverse group in the biosphere, with their genomes containing from a few genes to as many as hundreds of genes.
We can find them in every place on earth, where bacteria are able to live, even inside our bodies.
It is believed that one of the most saturated location of their occurrence is sea, with $9*10^{8}$ virions per milliliter of seawater at the surface and 70\% of bacteria infected. % http://mmbr.asm.org/content/64/1/69.full
Thanks to the high abundance of bacteriophages, their impact on shaping of the environment is outstandingly significant. % + circulating carbon?
% + they contribute to horizontal gene transfer via transduction and transformation

\section{Taxonomical classification}
% TODO: not sure if I want this section, but taxonomy of phages should probably be mentioned somewhere

\section{Structure of typical bacteriophage}
Given the high variety of bacteriophages, they come in a lot of different sizes and shapes.
Each bacteriophage consists of genetic information in form of DNA or RNA and capsid, a protein coat usually composited from higher number of the same protein unit.
In this section we will describe the most typical form of bacteriophages, which is abundantly found in nature.
As we can see on the picture, this bacteriophages body consists of head and tail.
Both of these part are created from proteins and in addition head contains DNA or RNA of the bacteriophages genome.
The tail is used to attach and to inject the genetic code to bacteria.
At the end of tail, there are proteins, which are able to bind to specific receptors on the surface of bacteria.
Thanks to this mechanism, bacteriophages tends to have high specificity in selection of they prey.

\section{Life cycle of bacteriophages}
Generally, there are 2 strategies, how bacteriophages try to secure their survival.
The first strategy is called lysogenic cycle and the second strategy is called lytic cycle.
Viruses tend to use this strategies in different proportions, but usually prefer to choose one of them.
The lysogenic cycle results in incorporations of phage genetic information into host DNA or creation of circular replicon in bacterial cytoplasm.
By this approach genetic material of phage inside host, called prophage, is duplicated together with the host genome and after cell division, both of daughter cells contain bacteriophages DNA.
In dependence on the following events, viral DNA can be released from hosts genome and start proliferation of new phages via the lytic cycle.
The lytic cycle is characterized by the lysis of bacterial cells membrane and their subsequent death.
It starts by injecting bacteriophages genome into bacteria.
After this step virus is not incorporated, but it compromises bacterial translation apparatus to produce more viruses.
Once enough virions have been produced, special viral proteins dissolve bacterial cell and virions are released into surrounding space.

\medskip
(picture of life cycle)
\medskip

\section{Potential usage of bacteriophages}
The ability to kill or alter behavior of bacteria with high specificity makes them valuable target for research.
Humankind tried to harness the power of phages since their discovery in year 1917.
This discovery is attributed to French-Canadian microbiologist Félix d'Hérelle, who experimented extensively with phages and introduced concept of phage therapy.
Unfortunately, after the discovery of antibiotics, research in this field suffered from insufficient funding and the development was significantly slowed down.
Nowadays, when humanity face the threat of multiresistant bacteria and phages could bring a new methods into the battle against pathogenic bacteria, the research in this field begin to flourish.
We can see the potential of their use in the field of food industry as the substances prolonging the life and improving the quality of our food,
in the field of pharmacy as medicines for bacterial infections and everywhere else, where we encounter the need to control the lives of microbial communities.

% \section{Goal of this work}
% \section{Current state of knowledge} !!!

\chapter{Computational methods of data retrieval and preprocessing}
In this chapter we will present our pipeline intended for data retrieval and preprocessing.
Since we used a considerable amount of third party libraries and tools, we will explain them in detail and we will clarify our reasons to use them.
We will also provide some basic statistics of data retrieved.

\section{Brief pipeline overview}
\section{downloading}
\section{genes identification}
\section{clustering}
\subsection{blast}
\subsection{global alignment}
\subsection{mcl}
\subsection{matrix creation}

\chapter{Machine learning methods of data analysis}
In this chapter we will explain methods used for analysis.

\section{Dimension reduction methods}
\subsection{univariate analysis}
\subsection{PCA}
\subsection{random forest/decission tree}
\section{interpretation}

\chapter{Results}
In this chapter we will provide results of our analysis and explain our findings.
% PCA