\chapter{Basics of molecular biology}
In this chapter we will present basic biological terms, which we are needed later in this work.
We will describe molecules performing functions in living organisms and we will define their representation in bioinformatics and computational biology.
We will elucidate relationships between these molecules, which will support our technique latter. 

\section{Biologically important molecules}

\subsection{Deoxyribonucleic acid}
Deoxyribonucleic acid (DNA) is a long chain of nucleotides. 
There are four types of nucleotides naturally occurring in DNA - adenine, cytosine, guanine and thymine. 
These bases are connected through phosphodiester bond, connection between the 3' carbon atom of one deoxyribose and the 5' carbon atom of the second deoxyribose.
The structure of DNA consists of two complementary strands coiled around each other in the form of double helix.
In this double helix adenine is paired with thymine and cytosine with guanine through hydrogen bonds.
In bioinformatics and computational biology we usually represent a base by its first letter - A for adenine, C for cytosine, G for guanine and T for thymine.
We usually store biological sequences in simple plain text files. 
Probably the most widely used format is FASTA format.
It can consist of one or multiple records, where each record starts with one line for record identifier, which is followed by possibly multiple lines of sequence written in direction from 5' to 3'.

\subsection{Ribonucleic acid}
Ribonucleic acid (RNA) is, similar to DNA, a polymeric molecule consisting of four types of nucleotides.
There are three main differences between these two molecules.
Firstly, the sugar-phosphate backbone of RNA contains ribose instead of deoxyribose.
This change in structure make RNA less stable as it is more prone hydrolysis.
Another distinctive feature of RNA is that it contains uracil instead of thymine.
Additionally, RNA appears in nature mostly as single-stranded molecule, whereas DNA is mostly double-stranded.
This characteristic allows RNA to form more complex structures and show enzymatic activity.
In bioinformatics, sequences of RNA are usually written, similarly as DNA, from 5' to 3', with the difference of U for uracil instead of T for thymine.

\subsection{Protein}
Proteins are polymers consisting of amino acids connected by peptide bond.
There are 20 basic proteinogenic amino acids encoded in DNA sequences and 2 proteinogenic aminoacids incorporated into proteins by unique mechanism.
Protein performs overwhelming majority of functions in living organisms.
Thanks to this characteristic, they will be of high importance in our analysis.
In bioinformatics protein sequences are represented as a string of 1-letter abbreviations of their amino acids. All proteinogenic amino acids with their 1-letter codes can be found in the Table (\ref{tab:amino})

\begin{table}
  \centering
    \begin{tabular}{ l  l  l  l }
      \hline
      amino acid & 1-letter code & amino acid & 1-letter code \\
      \hline  
      alanine & A & arginine & R \\
      asparagine & N & aspartic acid & D \\
      cysteine & C & glutamine & Q \\
      glutamic acid & E & glycine & G \\
      histidine & H & isoleucine & I \\
      leucine & L & lysine & K \\
      methionine & M & phenylalanine & F \\
      proline & P & serine & S \\
      threonine & T & tryptophan & W \\
      tyrosine & Y & valine & V \\
      selenocysteine & U & pyrrolysine & O \\
      \hline
    \end{tabular}
    \caption{Amino acids}
    \label{tab:amino}
\end{table}
 


\section{Central dogma of molecular biology}
Central dogma of molecular biology explains relationships between biological molecules described in previous sections.
Most organisms stores their genetic information in the form of DNA.
Some parts of DNA, usually called genes, are transcribed into RNA.
Afterwards, ribosome translate RNA sequence into protein based on RNA codon table.
Proteins are then folded into their natural 3D structure and prepared to realize their corresponding functions.

We based our work on central dogma of molecular biology and on assumption that genes will be effective features when predicting biological trait.




