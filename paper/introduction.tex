% v0.1
\chapter*{Introduction} % chapter* je necislovana kapitola
\addcontentsline{toc}{chapter}{Introduction} % rucne pridanie do obsahu
\markboth{Introduction}{Introduction} % vyriesenie hlaviciek

% TODO: 2 pages
Nowadays, excessive use of antibiotics leads to development of bacterial strain immune to conventional drugs.
The pace at which mankind is able to discover new medicaments for treating bacterial infection is slower than the pace of resistance growth.
Bacteriophages could introduce novel methods in the battle against bacteria.
Although, they are well known to mankind for over hundred year, we still do not know a lot about their mechanism of function.
Sometimes even basic information about their hosts are not found in databases.
In our work we will propose a method for predicting their host based only on their genomic sequence.
We will also show, how our method could provide a new insight on data about bacteriophages available online.

In the first chapter, we will describe basic terms from molecular biology as they will be important further in work.
We will also mention how biological data are stored in memory and we will explain relations between these molecules.

Second chapter will be about the central topic of this work - bacteriophages.
Details about their role in environment will be provided together with their structure, life cycle and concept, how we classify them.
Because their potential for use in medicine is high, we will outline their function as antibacterial agents.

In the third chapter we will provide detailed description of our software intended for predicting of their host.
The most common biological databases will be mentioned and also third party software used in this work will be described in details.
We will depict how we can assess a function of a protein, how to align different sequences and find out how similar they are and how to cluster data based on their comparability.

In fourth chapter, we will show how we analyzed data and how we interpreted results.
We will illustrate how principal component analysis and decision tree classifier work.
At the end, we will show quantified results of our work.