% v0.1
\chapter*{Introduction} % chapter* je necislovana kapitola
\addcontentsline{toc}{chapter}{Introduction} % rucne pridanie do obsahu
\markboth{Introduction}{Introduction} % vyriesenie hlaviciek

% TODO: 2 pages
Discovery of penicillin was one of the most important discovery of the last century.
Many similar antibiotic substances were discovered after penicillin proved itself as efficient weapon against bacteria.
In the year 1940, antibiotics were first prescribed as a medication to serious bacterial infection. 
Since then, antibiotics helped mankind to fight bacteria and saved millions of lives.

However, easy access and their excessive use led to overuse, resulting in bacterial strain immune to antibiotic treatment. 
In less than 10 years from introduction of penicillin to the market, penicillin resistant bacterial strain became significant problem.
The drugs discovered after penicillin followed similar trend.

In the year 1972, vancomycin was introduced to the market.
Resistance to this drug was so problematic to induce in laboratory conditions, it was believed it will not develop in real conditions.
Unfortunately, resistance to this drug was reported in 1979 and 1983.

Nowadays, resistance to nearly all developed antibiotics was observed.
The pace at which mankind is able to discover new medicaments for treating bacterial infections is slowing down, which, together with the relatively fast pace of resistance growth, could cause a severe problem. 

Bacteriophages, natural predators of bacteria, could introduce novel methods in the battle against bacterial infections.
Although, they are well known to mankind for over hundred year, research in this field was slowed down after introducing antibiotics to the public.
Therefore, we still do not know a lot of details about them.
Sometimes even basic information about their hosts are not found in databases.

In our work, we will propose a method for predicting their host based on their genomic sequence.
We will also show, how our method could provide a new insight on data about bacteriophages available online.

In the first chapter we will describe basic terms of the molecular biology as they will be important for the work.
We will put into context terms as nucleic acids and amino acids.
We will also picture how we can work with data representing these terms in the computer.
Bacteriophages, central topic of this work, will be introduced in this chapter as well.
We will provide details about their role in environment, structure, taxonomical classification, life cycle and potential as antibacterial agents.
At the end of this chapter, we will summarize approaches to predicting hosts of bacteriophages taken by other researchers.  

The second chapter will be devoted to bioinformatics algorithms used in the work.
Comprehensive description of the alignment problem will be provided.
We will define the problem and we will illustrate algorithms used for solving it.
Moreover, data clustering will be described in this chapter.

In the third chapter we will provide detailed description of the created software for prediction of bacteriophages hosts.
Because this software will use a lot of third party tools, we will describe them in details. 
Furthermore, the most common biological databases will be mentioned.
We will depict each step of our program and we will provide reasons for decisions made during programming. 

In fourth chapter we will show methods used for data analysis and how we interpreted results.
We will illustrate principal component analysis and decision tree classifier.
Results from these analyses will be provided.
We will also show how we evaluated these results to support method correctness.
At the end we will discuss limitations of this work and we will outline ideas for further work.