\section{Data clustering}
% 2-4pages = 3600-7200 characters
\subsection{Definition of problem}
\subsection{Markov Cluster Algorithm}
\subsection{Spectral clustering}
\subsection{Clustering in our pipeline}
Clustering is a problem where we want to determine closely related entities and put them in distinct groups, also called clusters.
In order to be able to distinguish which entity is more closely related to another, we usually use similarity matrix.
To create a similarity matrix we used tab separated file described in previous section.
In this section we will mention different approaches we used for clustering. 
% http://www.paccanarolab.org/scps/
\subsection{Markov Cluster Algorithm}
First approach we used was Markov Cluster Algorithm. 
We used already implemented version called mcl \cite{mcl}.
This software is often used in bioinformatics for clustering of sequences based on their similarity.
It has the capability to determine number of resulting clusters automatically and except input data, it requires only one parameter.
This parameter is called inflation and usually takes float number values from 1.2 to 5.0.
Smaller values results in less and bigger clusters.
In our work we needed as few clusters as possible, mainly due to number of phage records in our dataset.
In case of too many clusters we would have too many features for classifier and we would risk overfitting of our final models to training dataset.
For these reasons we used inflation value 1.2, which created 15017 clusters.

\subsection{Markov Cluster Algorithm with global alignment}
To get more accurate clustering we tried different approach, where we determined similarity score of alignments with needleman-wunsch algorithm. \cite{needleman-wunsch}
This algorithm is able to calculate best global alignment score of given protein sequences.
The difference between global alignment and local alignment is that former aligns sequences from end to end.
The resulting score was then used as input to mcl algorithm described in previous section.
By this approach we created 9176 final clusters.

\subsection{Spectral clustering}
Spectral clustering is different algorithm used for clustering.
In this work we tried SCPS implementation\cite{scps}.
Authors of this bioinformatics software declare quality of clusters quantified by a measure that combines sensitivity and specificity to be better by 28\% in comparison to mcl algorithm.
Unfortunately, memory required by this program was too high and we were not able to run it on our server with all input data. 

%-------------------------------------------------------------------------------
%--------end of deeply described parts--------------------------------------
%-------------------------------------------------------------------------------