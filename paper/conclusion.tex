% v0.1
\chapter*{Conclusion} % chapter* je necislovana kapitola
\addcontentsline{toc}{chapter}{Conclusion} % rucne pridanie do obsahu
\markboth{Conclusion}{Conclusion} % vyriesenie hlaviciek

% TODO: 1 page
The goal of this work was to discover relationships between functional elements of bacteriophage sequence and its host. 
We showed, we can predict bacteriophage host with a high accuracy based on these relationships


The goal of the thesis would be to discover relationship between functional
components of bacteriophage sequence and its potential host. Student would
prepare a database of known bacteriophages and their hosts. He would identify
their functional elements and cluster them according to genomic sequence
similarities. He would propose a classification method based on components
that are highly enriched in bacteriophages of specific hosts.
In this work we showed how to create models for predicting hosts of bacteriophages.
We started by introducing new biological terms as DNA, RNA and proteins.
Next, we explained relationships between these molecules ans we also showed how they are represented inside the computer.

- to by som poriadne prepracoval, v podstate len kopirujes Introduction. V Conclusion opisujes zavery svojej prace, chces sa tu vychvalit. Cize vymysleli sme redukovanu verziu fagov, na ktorej sa pomocou jednoduchej PCA dali odlisit zhluky fagov podla hostitela. Na to sme nabalili klasifikator, ktory to vedel velmi dobre klasifikovat.
- popis aj, ze podobny princip by sa dal pouzit na identifikaciu inych vlastnosti fagov
- ake je potencialne vyuzitie tohto prediktora
- ako sa da dalej rozsirit, napr. tie zmesne vzorky
- Mozno by tu bolo vhodne prehodit aj Limitations, uvidis

Because bacteriophages were the central topic of this work, we dedicated second chapter to them.
We showed that, despite of their microscopic bodies, their impact on the nature is enormous.
The lytic and lysogenic cycle were also explained as to understand these two cycles is important in order to understand the medical potential of bacteriophages.
We also mentioned structure of bacteriophage and taxonomical classification.
Next, we mentioned similar works to our work and we compared them.

In third chapter we described our program in details.
We described software used with emphasis on explanation of how this software works.
The detailed picture of our process was provided.
For every step we mentioned reasons, why we chose right this one.

Fourth chapter was about analyses performed with data and about results, which we got.
Firstly, we described principal component analysis and showed our data on the plot.
Secondly, decision tree classifier was presented with example of how we could visualize it.
At the end, evaluation of our models was carry out and the results were presented.