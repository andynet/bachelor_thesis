% v0.1
\chapter*{Conclusion} % chapter* je necislovana kapitola
\addcontentsline{toc}{chapter}{Conclusion} % rucne pridanie do obsahu
\markboth{Conclusion}{Conclusion} % vyriesenie hlaviciek

% TODO: 1 page
In this work we showed how to create models for predicting hosts of bacteriophages.
We started by introducing new biological terms as DNA, RNA and proteins.
Next, we explained relationships between these molecules ans we also showed how they are represented inside the computer.

Because bacteriophages were the central topic of this work, we dedicated second chapter to them.
We showed that, despite of their microscopic bodies, their impact on the nature is enormous.
The lytic and lysogenic cycle were also explained as to understand these two cycles is important in order to understand the medical potential of bacteriophages.
We also mentioned structure of bacteriophage and taxonomical classification.
Next, we mentioned similar works to our work and we compared them.

In third chapter we described our program in details.
We described software used with emphasis on explanation of how this software works.
The detailed picture of our process was provided.
For every step we mentioned reasons, why we chose right this one.

Fourth chapter was about analyses performed with data and about results, which we got.
Firstly, we described principal component analysis and showed our data on the plot.
Secondly, decision tree classifier was presented with example of how we could visualize it.
At the end, evaluation of our models was carry out and the results were presented.